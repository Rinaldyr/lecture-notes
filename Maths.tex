\documentclass{article}

\title{Mathematical Techniques for Computer Science - Notes}
\author{Harvey Hyatt}

\usepackage[a4paper, total={7in, 10in}]{geometry}
\usepackage{amsmath}
\usepackage{amsfonts}

\begin{document}
\maketitle

\section{Analytical Geometry in the Plane}

\paragraph{Points:}
Given points P and Q with coordinates
$\begin{pmatrix}p_1 \\ p_2\end{pmatrix}$
and
$\begin{pmatrix}q_1 \\ q_2\end{pmatrix}$,
their distance $d$ can be computed using pythagoras:
$$d = \sqrt{(q_1-p_1)^2+(q_2-p_2)^2}$$

\paragraph{Vectors:}
\begin{itemize}
\item
Pair
$\begin{pmatrix} v_1 \\ v_2 \end{pmatrix}$
can be a movement of the plane: every point
$P=\begin{pmatrix}p_1 \\ p_2\end{pmatrix}$
is shifted to
$P'=\begin{pmatrix}p_1+v_1 \\ p_2+v_2\end{pmatrix}$
\item Uppercase letters are used for points and lowercase with arrows for vectors.
\item A vector has length
$|\vec{v}|=\sqrt{(v_1)^2 + (v_2)^2}$
which is the distance each point travels under the movement described by $\vec{v}$.
\item A Vector of length 1 is a Unit Vector.
\item $\vec{0}=\begin{pmatrix}0\\0\end{pmatrix}$ is the null vector.
\item $\vec{v}/|\vec{v}|$ is the unit vector pointing in the same direction as $\vec{v}$.
\item The vector $\vec{PQ}$ that moves point P into point Q has coordinates (q1-p1, q2-p2).
\item All points X that can be reached from P by following some distance along the direction of $\vec{v}$ lie on the straight line $X = P + (s \cdot v)$.
\item This is a parametric representation of a line where the parameter is s.
\item If given two points P and Q in the plane then this defines a straight line $X = P + (s \cdot PQ)$.
\item If given two lines $X = P + (s \cdot v)$ and $Y = Q + (t\cdot w)$ their point of intersection satisfies $p1 + sv1 = q1 + tw1$ and $p2 + sv2 = q2 + tw2$. This can be solved using Gaussian Elimination.
\end{itemize}

\section{Geometry in 3 Dimensions}

\paragraph{Planes}
The parametric representation of a plane has the form $X=P+s\cdot \vec{v}+t\cdot \vec{w}$ where P is a point in space and $\vec{v}$ and $\vec{w}$ are vectors (neither of which are null). $\vec{w}$ must not point in the same (or opposite) direction as $\vec{v}$, otherwise it is just a line.
\vspace{1mm}\\
Three points P, Q and R which are not all on the same line determine a plane $X=P+s\cdot \vec{PQ}+t\cdot \vec{PR}$.

\paragraph{Intersection Tasks}
\begin{itemize}
\item \textbf{Test whether a point lies on a line or plane} For point Q: $X = P + (s \cdot v)$ therefore $P + (s \cdot v) = Q$.
\item \textbf{Find the intersection point between lines/planes} Simply set the two equations as equal to each other.
\end{itemize}
Solve using Gaussian Elimination.

\paragraph{Two-point description of a line} Two points P and Q determine a line
$$X = P + s\cdot \vec{PQ}$$
This can be rewritten as
\begin{align*}
X &= P+s\cdot (Q-P)\\
&= P + s\cdot Q-s\cdot P\\
&= (1-s)\cdot P+s\cdot Q
\end{align*}
This can also be done with a plane using 3 points.

\paragraph{Vector spaces}
So we have a null vector, we can add two vectors, and we can multiply vectors with a scalar.
Any structure that satisfies the laws of vector algebra and carries these two operations is called a \textbf{vector space}.
\vspace{1mm}\\
The idea of a vector space is more complex than just 3-dimensional movements. Scalars can come from any field $\mathbb{F}$. We will call one theoretical field $GF(2)$. One then speaks of a ``vector space over $\mathbb{F}$''.

\paragraph{Subspaces}
If $\vec{v}$ is an element of some vector space then we can generate a \textbf{subspace} (a \textbf{sub-vector space}) by considering all vectors of the form $s\cdot \vec{v}$. Another subspace would be all expressions of the form $s\cdot \vec{v} + t\cdot \vec{w}$.
\vspace{1mm}\\
Another way of looking at parametric representations is that we pick a point and allow all movements from a subspace to act on this point. This leads to an \textbf{affine subspace}. In other words, lines and planes are affine subspaces of 2D/3D.

\paragraph{Bases}
If two generators point in the same direction then one of them is redundant and can be removed. A set of generators that includes a redundant generator is said to be \textbf{linearly dependent}. A set of generators that can not be made any smaller is a \textbf{basis} of the subspace. The number of elements of a basis is the \textbf{dimension} of the subspace.
\vspace{1mm}\\
If we start with some set of generators and we are not sure if any of them are redundant, we can write the vectors as rows into a matrix and run Gaussian elimination. The rows of the resulting echelon form will be a basis for the subspace. Any redundancy will show itself as rows consisting entirely of zeros.

\paragraph{Codes}
Subspaces in $GF(2)^n$ are used as linear \textbf{codes} in coding theory. This is coding to spot and correct errors in data, not programming or cryptography.

\end{document}
